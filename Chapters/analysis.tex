\chapter{Analysis (Generic title)}
\label{chap:analysis}

This chapter describes the practical and theoretical foundation of your project. Basically, there are two aspects you should focus on, your research topic, and related work (literature and projects).

\section{Research topic (generic title)}

Here you will describe the thesis topic in sufficient detail to work out the details of your project, so that the reader gets a perfectly clear picture of the settings of your project.  It is important to define your scope, and perhaps narrow down a broad subject. Also, if there are such, describe constraints and requirements you need to follow. If your work is part of a larger project, or if you are cooperating with an external company or research institute, this is the place to tell the reader about that.

\lipsum[61-67]

\section{Related work (generic title)}

It is important that you relate your work to relevant research and projects, and base your work solidly on existing literature. 
In particular, you must highlight related work that are directly relevant for your project, for instances if you want to extend earlier research, or to use specific results from other projects.

You should also discuss alternative research methods that have been used to research similar problems.

\lipsum[30-36]

\section{Methods (generic title)}
\lipsum[30-36]

\section{Tools (generic title)}
\lipsum[30-36]

\section{Summary (Optional)}

Sometimes, in particular when the chapters are quite long, they are ended with short summaries.   
