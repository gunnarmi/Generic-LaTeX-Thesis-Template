\chapter{How to use this template}
\label{chap:how-to} 

This document is a generic thesis template covering both structure, content, and layout, to be used as a starting point for bachelor's, master's, and PhD theses in any field. 
It is implemented in the \gls{latex} typesetting markup language, which in particular is suitable 
for documents including \gls{maths} 
and scientific notation, and is considered the gold standard in natural sciences, however commonly used in other settings.

The template is implemented as a document class, 
\texttt{thesistemplate}, with the following options:

\begin{compactdesc}
\item [\texttt{draft}] Turns off graphics rendering and micro-formatting stuff, and turns on todo notes. Affects page breaking.
\item [\texttt{cover}] Custom cover page, must be provided as an Layout/coverpage.PDF in A4 size. Otherwise a standard HiØ (Høgskolen i Østfold) cover is generated.
\item [\texttt{word}] Word-like paragraph formatting, no indentations, air between paragraphs, and ragged right margin. Otherwise standard \LaTeX\ paragraph formatting (as you also find it in most books, papers, and formal documents).
\item [\texttt{sans}] Sans serif fonts, otherwise standard \LaTeX\ font (Computer Modern).
\end{compactdesc}

First of all, it outlines a recommended structure of the thesis, and provides guidelines for the content of each part.
In addition, it serves as a concrete example of how this can be accomplished by using \LaTeX. Use the template by gradually populating the files with your own content, and perhaps modify the structure to suit your project. The template is designed to be rather self-explanatory, and all of the features you need are present somewhere in the source code, so you will come a long way by cutting and pasting.

It is assumed that the user has (or provides herself with) some basic knowledge of \LaTeX. There are numerous good tutorials online\footnote{\url{latex-project.org} is a good starting point}, but I warmly recommend the original documentation: ``\LaTeX: A Document Preparation System'' (Figure \ref{fig:lamport})  \cite{lamport94ldp}. \LaTeX\ is basically a collection of macros written in \TeX.
This system, which is a low level tool for digital typesetting, is known for producing scientific documents of unprecedented quality. It was developed by Donald Knuth, one of the giants in computer science.

\begin{figure}[!htbp]
\centering 
    \includegraphics[width=0.3\textwidth]{Graphics/lamport}
    \caption{The \LaTeX\ ``bible'', 2. edition \label{fig:lamport}}
\end{figure}


The template is intended to use with Overleaf\footnote{\url{overleaf.com}}, where it works out of the box.

However, it can be used locally on your machine, using one of the many good \LaTeX\ editors, and there are many possibilities for the Emacs users\footnote{\url{www.gnu.org/software/emacs}}. Personally, I use {\em texmaker}\footnote{\url{www.xm1math.net/texmaker/}}
for OSX, and 
{\em Kile}\footnote{\url{kile.sourceforge.net}}
for Ubuntu. MS users may try {\em WinEDT}\footnote{\url{www.winedt.com}}.


\section{Language settings}

The template is using the \texttt{polyglossia} package, for enabling easy use of different languages, including native UTF-8 support. You may make multi-language documents \footnote{\url{www.overleaf.com/learn/latex/Multilingual_typesetting_on_Overleaf_using_polyglossia_and_fontspec}}. You set the main language for instance like this (in the preamble, before \verb|\begin{document}|):

\verb|\setdefaultlanguage[variant=bokmal]{norwegian}|

See \cite[p.~6]{polyglossia}\footnote{\url{https://ctan.uib.no/macros/unicodetex/latex/polyglossia/polyglossia.pdf}} for supported languages, including variants.

\section{Chapters/Sections/Paragraphs}

As any document, the template is a collection of chapters, sections, subsections, subsubsections and paragraphs. By default subsubsections and paragraphs are not numbered or included in the table of contents. 

These structural parts contain
plain text and/or graphical elements like figures and tables. 
Plain text is commonly structured by {\em paragraphs}. Paragraphs are generated by one or more {\em empty lines}\footnote{Correspondingly, when there are two or more       consecutive          whitespaces in the text, these will treated as one single whitespace.}. 

{\em Chapters} have {\em sections}, which may contain {\em subsections}, and the next level is {\em subsubsection}.
Finally we have a special type of {\em paragraph}.

Please note that it is considered good practice to have some text before you go to a lower section level, for instance, do not go directly from a chapter to a section, and, in general, do not skip levels, like going from a section to a subsubsection.

Below follows examples of all these constructs.


\section{Section} 
This is a section. \lipsum[10-12]
\subsection{Sub section} 
This is a sub section. \lipsum[13-14]
\subsubsection{Sub sub section} 
This is a sub sub section. \lipsum[15-16]
\paragraph{Paragraph} 
This is a titled paragraph.
Please note the difference between the standard paragraphs produced by blank lines, and this type, which is the lowest level of elements with titles.

\lipsum[17-18]

\section{Figures, tables, equations, etc.}

Please see the source code (\texttt{how-to.tex}) for details on the implementation of these elements.

In general, figures and tables shall be numbered, and have a caption. Numbered figures and tables {\em must} be referenced at least once in the text. Equations and similar elements should also be numbered, but they are not always referred to.

Figures, tables, equations, and similar constructs are so-called {\em floats}. This means that \LaTeX\ will place them in a position that is ``best'', taking many aspects into consideration. The result is that the elements may not be positioned exactly where the author wants (in particular when you have many floats near each other, like in text you are reading now), and novice user may find this frustrating (see Section \ref{sec:bestpractise}). Remember to include empty lines in the source before and after a float.

\begin{figure}[!htbp]
  \begin{center}
    \subfigure[input]{\label{fig:aust}\includegraphics[width=2.5in]{Graphics/introaustralia}}
    \subfigure[output]{\label{fig:approx}\includegraphics[width=2.25in]{Graphics/australia_approx}}
  \end{center}
  \caption{Input and result from running the Douglas-Peucker line simplification algorithm (from  \cite{kjeldsen05cor})}
  \label{fig:dpaustralia}
\end{figure}

Figures are most often produced from files in common graphics formats (like PDF, png, jpg, etc). You can use a single image file, as in Figure \ref{fig:lamport}, or you can combine several images, see Figure \ref{fig:dpaustralia}, consisting of Figures \ref{fig:aust} and \ref{fig:approx}.

The template provides two short-hand commands for including graphics. Figure \ref{fig:floatingfigure} is produced with the \texttt{floatingfigure} command, which let \LaTeX\ to treat it like a floating object, placing it at at the ``best" location. The first argument is the filename, the second gives the width as a fraction of the text width, and the last sets the caption.

\floatingfigure{Graphics/introaustralia}{0.6}{A figure produced by the \texttt{floatingfigure} command}
\label{fig:floatingfigure}

Despite of beeing typed directly after the floating figure, note that this text may for example appear prior to the figure in the PDF.

Ut lectus lectus, ultricies sit amet, semper eget, laoreet non, ante. Proin at massa quisnunc rhoncus mattis. Aliquam lorem. Curabitur pharetra dui at neque. Aliquam eu tellus. Aenean tempus, felis vitae vulputate iaculis, est dolor faucibus urna, in viverra wisi nequenon risus. Fusce vel dolor nec sapien pretium nonummy. Integer faucibus massa ac nullaornare venenatis. Nulla quis sapien. Sed tortor. Phasellus eget mi. Cras nunc.

The \texttt{fixedfigure} command forces the graphics to be placed exactly where it is located in the text
\footnote{Be careful, this could cause some strange effects.}, 
as in Figure \ref{fig:fixedfigure}.

\fixedfigure{Graphics/introaustralia}{0.6}{A figure produced by the \texttt{fixedfigure} command}
\label{fig:fixedfigure}

After inserting a fixed figure, this text should appear directly after the figure in the PDF.

Ut lectus lectus, ultricies sit amet, semper eget, laoreet non, ante. Proin at massa quisnunc rhoncus mattis. Aliquam lorem. Curabitur pharetra dui at neque. Aliquam eu tellus. Aenean tempus, felis vitae vulputate iaculis, est dolor faucibus urna, in viverra wisi nequenon risus. Fusce vel dolor nec sapien pretium nonummy. Integer faucibus massa ac nullaornare venenatis. Nulla quis sapien. Sed tortor. Phasellus eget mi. Cras nunc.

Ut lectus lectus, ultricies sit amet, semper eget, laoreet non, ante. Proin at massa quisnunc rhoncus mattis. Aliquam lorem. Curabitur pharetra dui at neque. Aliquam eu tellus. Aenean tempus, felis vitae vulputate iaculis, est dolor faucibus urna, in viverra wisi nequenon risus. Fusce vel dolor nec sapien pretium nonummy. Integer faucibus massa ac nullaornare venenatis. Nulla quis sapien. Sed tortor. Phasellus eget mi. Cras nunc.

Ut lectus lectus, ultricies sit amet, semper eget, laoreet non, ante. Proin at massa quisnunc rhoncus mattis. Aliquam lorem. Curabitur pharetra dui at neque. Aliquam eu tellus. Aenean tempus, felis vitae vulputate iaculis, est dolor faucibus urna, in viverra wisi nequenon risus. Fusce vel dolor nec sapien pretium nonummy. Integer faucibus massa ac nullaornare venenatis. Nulla quis sapien. Sed tortor. Phasellus eget mi. Cras nunc.
Ut lectus lectus, ultricies sit amet, semper eget, laoreet non, ante. Proin at massa quisnunc rhoncus mattis. Aliquam lorem. Curabitur pharetra dui at neque. Aliquam eu tellus. Aenean tempus, felis vitae vulputate iaculis, est dolor faucibus urna, in viverra wisi nequenon risus. Fusce vel dolor nec sapien pretium nonummy. Integer faucibus massa ac nullaornare venenatis. Nulla quis sapien. Sed tortor. Phasellus eget mi. Cras nunc.

If you want to wrap text around an illustration, the \texttt{wrapfig} package can be used
\footnote{The environment should be placed so as to not run over a page break, that could cause that the figure overlaps the text.}
, as demonstrated in Figures \ref{fig:wrap-inner} and \ref{fig:wrap-outer}.

\begin{wrapfigure}{O}{0.4\textwidth}
  \centering 
  \includegraphics[width=0.38\textwidth]{Graphics/introaustralia}
  \caption{Wrapped figure placed near the outer margin}
  \label{fig:wrap-outer}
\end{wrapfigure}

Ut lectus lectus, ultricies sit amet, semper eget, laoreet non, ante. Proin at massa quisnunc rhoncus mattis. Aliquam lorem. Curabitur pharetra dui at neque. Aliquam eu tellus. Aenean tempus, felis vitae vulputate iaculis, est dolor faucibus urna, in viverra wisi nequenon risus. Fusce vel dolor nec sapien pretium nonummy. Integer faucibus massa ac nullaornare venenatis. Nulla quis sapien. Sed tortor. Phasellus eget mi. Cras nunc.

Ut lectus lectus, ultricies sit amet, semper eget, laoreet non, ante. Proin at massa quisnunc rhoncus mattis. Aliquam lorem. Curabitur pharetra dui at neque. Aliquam eu tellus. Aenean tempus, felis vitae vulputate iaculis, est dolor faucibus urna, in viverra wisi nequenon risus. Fusce vel dolor nec sapien pretium nonummy. Integer faucibus massa ac nullaornare venenatis. Nulla quis sapien. Sed tortor. Phasellus eget mi. Cras nunc.

\begin{wrapfigure}{I}{0.4\textwidth}
  \centering 
  \includegraphics[width=0.38\textwidth]{Graphics/introaustralia}
  \caption{Wrapped figure placed near the inner margin}
  \label{fig:wrap-inner}
\end{wrapfigure}

Aenean tempus, felis vitae vulputate iaculis, est dolor faucibus urna, in viverra wisi nequenon risus. Fusce vel dolor nec sapien pretium nonummy. Integer faucibus massa ac nullaornare venenatis. Nulla quis sapien. Sed tortor. Phasellus eget mi. Cras nunc.

Aenean tempus, felis vitae vulputate iaculis, est dolor faucibus urna, in viverra wisi nequenon risus. Fusce vel dolor nec sapien pretium nonummy. Integer faucibus massa ac nullaornare venenatis. Nulla quis sapien. Sed tortor. Phasellus eget mi. Cras nunc.

Aenean tempus, felis vitae vulputate iaculis, est dolor faucibus urna, in viverra wisi nequenon risus. Fusce vel dolor nec sapien pretium nonummy. Integer faucibus massa ac nullaornare venenatis. Nulla quis sapien. Sed tortor. Phasellus eget mi. Cras nunc.

Tables have a relatively steep learning curve.
Still, simple tables, like Table \ref{tab:simple}, are relatively easy to make.
A more complex example is demonstrated in Table \ref{tab:complex}\footnote{An easy workaround is to generate the table in a spread sheet program, export to an image or PDF file and include as graphics.}.

\begin{table}[!htbp]
\centering
\begin{tabular}{c|c|c}
X &  & \\
\hline
& X & \\
\hline
 &  & X \\
\end{tabular}
\caption{Simple table}
\label{tab:simple}
\end{table}

\begin{table}[!htbp]
\centering
\resizebox{0.6\textwidth}{!}{%
\begin{tabular}{|>{\bfseries}c||p{1.0em}|p{1.0em}|p{1.0em}|p{1.0em} l|}      \hline
\textbf{Combination} &\multicolumn{5}{l|}{\textbf{Included Optional Steps}}\\\cline{2-6}
   & \textbf{1}&\textbf{2}&\textbf{3}&\textbf{4}&\\\hline\hline
 1 & X & & & &    \\\hline
13 &  &  & X & X &\\\hline
14 &  &  &  & X & \\\hline\hline
15 & \multicolumn{4}{l}{Nano Particles Deposited, Not Sintered}&\\\hline
16 & \multicolumn{4}{l}{Only Grinded Wafer 1, No Particles Deposited, Not Sintered}&\\\hline
17 & \multicolumn{4}{l}{Only Grinded Wafer 2, No Particles Deposited, Not Sintered}&\\\hline
\end{tabular}}
\caption{Complex table}
\label{tab:complex}
\end{table}

Within mathematics and natural sciences there is a common belief that \LaTeX\ is unrivaled when it comes to typesetting formulas, equations, and complex specialized notation, as the following examples demonstrate.

You can have inline equations, like this: $ \alpha = \beta \gamma \delta $, or you can typeset them as numbered floats, as in Equation \ref{eq:abc}.
\begin{equation}
  \alpha = \beta \gamma \delta
  \label{eq:abc}
\end{equation}

Equation \ref{eq:mom-inert} is a bit more complicated:

\begin{equation}
  I_{zz} = \int_{-b/2}^{b/2} \int_{-h/2}^{h/2} y^2 dy dx = \frac{b h^3}{12}
  \label{eq:mom-inert}
\end{equation}

There are loads of special characters, like $\approx$, $\pm$,
$\times$, $\div$, $\propto$, $\leq$, $\geq$, $\ll$, $\gg$, $\neq$,
$\nabla$, $\Re$, $\Im$, $\flat$, $\sharp$, $\partial$, $\infty$, and $\heartsuit$.


See the next section for a complete example of a mathematical proof.

\section{Proof of the Area of a Circle Formula}

Here follows a complete proof of a theorem.

\newtheorem{prf}{Theorem}

\begin{prf}
The area of circle with radius $r$ is $\pi r^2$.
\end{prf}

\noindent \textbf{Proof:} The equation of a circle centered at the
origin is

$$
x^2 + y^2 = r^2,
$$

\noindent where $r$ is the radius.  We  write $y$ in terms of the
variable $x$ and the constant $r$:

$$
\frac{x^2}{r^2} + \frac{y^2}{r^2} = 1
$$
$$
\frac{y}{r} = \sqrt{1-\frac{x^2}{r^2}}
$$
$$
y= r\sqrt {1-\frac{x^2}{r^2}}
$$

By symmetry, the area of a circle centered at the origin is four
times the area of the circle between $(0,0)$ and $(r, 0)$ above the
$x$-axis.  We can integrate to find the area ($A$):

$$
A = 4r\int_0^r \sqrt {1-\frac{x^2}{r^2}}\, dx
$$

To evaluate the antiderivative of $\displaystyle\sqrt
{1-\frac{x^2}{r^2}}$, we make the substitutions:

$$
x = r \sin \theta
$$
$$
\theta = \arcsin \frac{x}{r}
$$
$$
dx = r\cos \theta\, d\theta
$$

Thus, our integral becomes:

$$
A=4r\int_0^r \sqrt {1-\frac{x^2}{r^2}}\, dx = 4r\int_0^{\pi/2}
r\sqrt{1-\sin^2 \theta} \cos \theta\, d\theta
$$

 We can use the trigonometric identity $1 - \sin^2 \theta = \cos^2 \theta$:

$$
A=4r\int_0^{\pi/2} r\sqrt{1-\sin^2 \theta} \cos \theta\, d\theta=
4r^2\int_0^{\pi/2} \cos^2 \theta\, d\theta
$$

We then apply $\cos^2 \theta = \frac{1}{2}(1 + \cos 2\theta)$:

\begin{eqnarray*}
4r^2\int_0^{\pi/2} \cos^2 \theta\, d\theta &=& 4r^2\int_0^{\pi/2}  \frac{1}{2}(1 + \cos 2\theta) \,d\theta\\
& = & {2r^2\theta}\Bigg{|}_0^{\pi/2} + 2r^2\int_0^{\pi/2} \cos 2\theta \,d\theta\\
                                  & = & \pi r^2 + 2r^2(\sin2\theta)\Bigg{|}_0^{\pi/2}\\
                                  & = & \pi r^2
\end{eqnarray*}

Thus, the area of a circle with radius $r$ is $\pi r^2$.\hfill$\blacksquare$

\section{Listings and other {\em environments}}

You can apply specialized layout by using  {\em environments}. Environments are constructed like this:

\begin{lstlisting}[float=htpb]
\begin{some-environment}
The text and other contents goes here
\end{some-environment}
\end{lstlisting}

The most common environments are the following three different list types\footnote{Here we are using compact versions of the standard lists, which tend to produce to much ``air''}. First, the bullet list:

\begin{compactitem}
\item First item
\item Second item
\item Third item
\end{compactitem}
Then, the enumerated list:
\begin{compactenum}
\item First item
\item Second item
\item Third item
\end{compactenum}
And finally the decription list:
\begin{compactdesc}
\item [First item] First description \lipsum[5]
\item [Second item] Second description
\item [Third item] Third description
\end{compactdesc}

Needless to say, as anything else in \LaTeX, these lists can be customized to your liking.

\section{Source code}
\label{sec:sourcecode} 

Large chunks of code should be placed in an appendix, but smaller pieces may be listed in the main part. We here demonstrate two ways of doing this.

In Listing \ref{list:hanoi} we have included code from a separate file.

\lstinputlisting[caption=Recursive solution of Towers of Hanoi,label=list:hanoi, language=Java,float=htpb]{Code/hanoi.java}
\index{Recursion|see{Recursion}}
\index{Towers of Hanoi}

Listing \ref{list:hanoi2} is generated by copying and pasting from the same file.

\begin{lstlisting}[caption=Core of the recursive solution of Towers of Hanoi,label=list:hanoi2,language=Java,float=htpb]
// move n smallest discs from one pole to another, using the temp pole
 public static void hanoi(int n, String from, String temp, String to) {
     if (n == 0) return;
     hanoi(n-1, from, to, temp);
     System.out.println(''Move disc '' +n+ '' from '' +from+ '' to '' +to);
     hanoi(n-1, temp, from, to);
}
\end{lstlisting}

\section{Cross-references and bibliography}

As mentioned earlier, all non-text elements should be numbered, and should be referenced to at least once in the text.
This is what is called {\em cross-referencing}, and is easily accomplished.
First we need to attach a label to the element: \verb|\label{type:name}|. Then we use this label in the reference: \verb|\ref{type:name}|. The reference is only a number, so we usually add the element type as a capitalized prefix, for instance like his: \verb|Figure \ref{fig:lamport}}|, which produces: Figure \ref{fig:lamport}.

When you need a reference to an item in your bibliography (books, articles, web sites etc), you need to use one or more ``database'' files, which are plain texts files with bibliography items formatted according to certain rules. These files have the extension \texttt{.bib}, and must be included in the the preamble in the main file.
An example of a correctly formatted bibliography item is found in \ref{list:BibLaTeX}.

\begin{lstlisting}[caption=BibLaTeX entry,label=list:BibLaTeX,language=Tex,float=!htbp]
 @book{perelman97mht,
 author = {Perelman, Leslie and Barrett, Edward},
 title = {{The Mayfield Handbook of Technical and Scientific Writing}},
 year = {1997},
 edition = {1},
 publisher = {McGraw-Hill, Inc.},
 address = {New York, NY, USA},
} 
\end{lstlisting} 


This format is called  {\em BibTeX}, and all the academic search engines, including
 {\em Google Scholar}, export to this format.
When referencing, you use this command: \verb| \cite{perelman97mht}|
and you get:  \cite{perelman97mht}. If you want to specify page(s), use \verb|\cite[p. XX]{perelman97mht}|, which results in \cite[p. XX]{perelman97mht}.
 
To build the bibliography, run \verb|\printbibliography[heading=bibintoc]|, which also places an entry in the table of contents.
BibLaTeX generates the  bibliography only from the references in the document (and not from all the items in the \texttt{.bib} files).

The different scientific communities have their own guidelines for how to format the entries in the bibliography, and how to format the references in the document.
You decide which style to use with the command \verb|\citationstyle{ieee}| in the preamble in the main file.

There are many options for creating specialized bibliography, for instance multiple bibliographies\footnote{\url{www.overleaf.com/learn/latex/bibliography_management_with_biblatex}}.

There are several tools for creating and maintaining bibliography databases that export BibLaTeX files, both standalone programs, plugins to editors and browsers, and cloud based services\footnote{For instance, check out \url{zotero.org}}. I recommend to use such reference managers, in particular the online versions which, when shared, enable collaborative writing.

\section{Glossary and Index}

You may also make a glossary page, where the terms have descriptions and lists of pages they occur on. You need to provide the terms and their descriptions in the preamble, for instance in a separate file, see \verb|glossary.tex|. An entry looks like this:

\begin{lstlisting}[float,float=!htbp]
\newglossaryentry{maths}
{
    name=mathematics,
    description={Mathematics is what mathematicians do}
}
\end{lstlisting}

The terms are marked in the text like this:

\verb|\gls{latex}|

which generates
\gls{latex}. The glossary is produced by including \verb|\\printglossaries| in the main file, and run \verb|makeglossaries| when compiling. 

The result in the glossary will be something like this:  

``mathematics Mathematics is what mathematicians do. 1, 29''.

An index is made by marking words, for instance \index{IMRAD}, like this

\verb|\index{IMRAD}|

The index page is then produced by including \verb|\printindex| in the main file, and running the following sequence\footnote{Overleaf does this automatically when compiling.}:

\begin{lstlisting}[float,float=!htbp]
xelatex main
makeindex main
xelatex main
\end{lstlisting}

The entry in the index page will then be:  

``IMRAD, 48''.

\section{Fonts}

Fore every  \LaTeX\ distribution, there is a default set of fonts.
It is possible to customize this  setup.
Normally, there is no need for this.
Nevertheless, to change the fonts for the whole document, use the following command in the preamble:

\verb|\usepackage[lining,light,default]{sourcesanspro}|
with the name of the typeface and the appropriate options\footnote{For a list of available fonts, see for instance \url{www.overleaf.com/learn/latex/font_typefaces\#Reference_guide}}.

However, size, weight, style and typeface may be manipulated using standard commands.

\paragraph{Font size}

First of all, you decide in the preamble the default font size for your document. The \texttt{documentclass} takes the parameters \texttt{10pt},  \texttt{11pt},  or \texttt{12pt}. Locally, you can change the style by these commands, that resizes the font {\em relatively} to the default size:

    \tiny tiny
    
    \scriptsize scriptsize
    
    \footnotesize footnotesize
    
    \small small
    
    \normalsize Default: normalsize
    
    \large  large
    
    \Large Large
    
    \LARGE LARGE
    
    \huge huge
    
    \Huge  Huge
    
    \normalsize 
    
\paragraph{Font weight (Font series)}
You can locally change the font weight:
    
    \textmd{Default: Medium}
    
    \textbf{Bold}
    
    
\paragraph{Font style (Font shape)}
The font style may also be locally changed:
    
    \textup{Default: Normal (Upright/Roman)}
    
    \textit{Italic}
    
    \textsl{Slanted}
    
    \textsc{Small caps}
    
\paragraph{Font family}
Here is how you locally change font family:
 \begin{compactitem}
  \item \textrm{Default: Roman (serif)}
  \item \textsf{Sans serif}
  \item \textit{Italic}
  \item \textbf{Bold}
  \item \textmd{Medium}
  \item \textsl{Slanted}
  \item \textup{Upright}
  \item \textsc{Small caps}
  \item \textbf{Bold}
 \end{compactitem}   
    
    
\section{To-do notes}
\label{sec:miscellaneous}

Many find it useful to use todo-notes. You can place it inline, with

\verb|\todo[inline]{This is an inline note.}|

\todo[inline]{This is an inline note.}
Or, if you prefer, in the margin with 

\verb|\todo[inline]{This is a margin note.}|

\todo{This is a margin note.} The todo notes will only be visible when using the \texttt{draft} option in the document class.

\section{Compilation}


Making documents with \LaTeX\ is basically like writing software. 
This document is produced by compiling a collection of files, all in the same folder.
There is a top level file called 
\texttt{main.tex}, which contains commands that decide format, layout etc., or in other words, the {\em style}\footnote{Think of HTML and stylesheets \dots guess where that idea came from \dots}. It also includes the files containing the actual text (in general one file for each chapter).

When using Overleaf, the compilation is done automatically when saving a file in the project, or when issuing the \texttt{Recompile} command.


When using the template locally, you may compile it with the appropriate commands in your IDE, or do it from the command line. To generate a PDF file, go to the document folder, and issue the following commands: 

\begin{compactenum}
\item \verb|xelatex main| (runs through the document, setting up various help files and generates a preliminary PDF)
\item \verb|biber main| (generates the bibliography from the citations)
\item \verb|xelatex main| (includes the bibliography)
\item And finally yet another \verb|xelatex main| to get all cross-references and citations correct.
\end{compactenum}


This process produces a PDF file,
\texttt{main.PDF}. When printing this particular document, remember to select the double page option.

\section{Quotes and quotations}

\LaTeX\ is picky on quotation marks, to produce correct start and end marks you should type \verb|``a quote''|, which produces ``a quote''. To be on the safe side, use the \verb|\say| command, which also enables nested quotes, like this: \say{A quotation may have \say{nested} quotations}.

Quotation marks are used when you want inline quotes. However, longer quotations should be made with the ``quote'' (or ``quotation'') environment:
\begin{quote} 
    \lipsum[3-4]
\end{quote}

Remember to always cite the source of a quotation like this.

\subsection{Troubleshooting}

As you may experience when compiling the source code, there might be syntax errors, missing files etc. to be fixed. In Overleaf, error and warning messages are displayed as colored margin markers in the source text, and in the PDF pane. Red flags are indicating errors that Overleaf/\LaTeX\ is not able to fix, thus not able to generate the PDF. These needs to be fixed by you. Overleaf tries to repair minor errors, giving red flags, but still generates the PDF. These errors should also be corrected.

You should also try to fix the orange warnings, but they are not critical. Blue warnings may be ignored, they mostly complain about ``overfull'' and ``underfull'' boxes, which in most cases do not affect the final document in any perceivable way.

If stuck, consult the Overleaf troubleshooting pages\footnote{\url{www.overleaf.com/learn/latex/Errors} and
\url{www.overleaf.com/learn/latex/Questions/Tips_and_Tricks_for_Troubleshooting_LaTeX}}. In some rare cases you need to delete all temporary help files, if so, go to the bottom of the error/warning pane and choose ``Clear cache''. If you run the project locally, this is done by deleting all \texttt{main.*} files, except \texttt{main.tex}.

However, sometimes it can be really tricky to find the source for an error. You should typically search backwards from where the error occurs. Listing \ref{list:latexerror} is an example from compiling this document, where the error is 
misspelling of the \LaTeX\  macro (should be \verb|\LaTeX|, not \verb|\LateX|). The key error message is ``\verb|! Undefined control sequence|'', followed by a quote of the line where the error has occurred, along with the line number. The name of current file is found a couple of lines above: ``\verb|(./how-to.tex|''.

\begin{lstlisting}[float=htpb, caption=\LaTeX\ error output,label=list:latexerror]
...

Overfull \hbox (6.0pt too wide) in paragraph at lines 43--44
[][][][][][]

Underfull \hbox (badness 10000) in paragraph at lines 43--44

) (./conclusion.tex [20]
Chapter 7.
[21]) (./main.bbl [22]) [23] [24]
No file main.ind.
(./how-to.tex
Appendix A.
[25]
! Undefined control sequence.
l.48 misspelling of the  \LateX
                         \  macro (should be \verb|\La...

? 
} 
\end{lstlisting} 



\section{Best practice}
\label{sec:bestpractise}

\begin{compactitem}
\item First: Focus on {\em content} and {\em structure} 
\item Later: Decide on layout and style
\item Use mostly the default settings
\item If you need special functionality, look for packages covering your needs
\item If you do not find suitable packages, make your own macros
\item Learn by 1) asking fellow students, 2) google and cut'n paste, and 3) by sending me an email or come to my office
\item Compile frequently
\item Commit frequently to your versioning system\footnote{SVN is a good choice, or use any of the many free online services.}
\item Run spell checks when things start to get complete\footnote{Most editors provide built-in spell check functionality (which ignores the markup commands).  On Linux platforms you have the ispell and aspell command line tools which can be configured for \LaTeX. There are also stand-alone tools around.}
\item When the document absolutely complete: perform minor fine-tuning (typically to sort out bad placements of floats). Remember that every fix you apply may affect the subsequent layout.
\end{compactitem}



