\chapter{Results / Testing / Evaluation (Generic Title)}
\label{chap:evaluation} 

You now present the outcomes of the actual research work described in the previous chapter, as suggested in \parencite{perelman97mht}:


\begin{quotation}
In the results section of a report, describe all appropriate information produced by the research procedures. Simply present data and estimates of their accuracy. Save the explanation and interpretation of these findings for the discussion section, which usually follows the results section. In short documents, however, the results and discussion sections may be combined into a single section.

Results sections make extensive use of graphs and figures to present data effectively. Order information by its importance to your audience's purpose in reading the document. State all significant findings in the text, referring to tables and graphs displaying all significant data. If the study has produced a large amount of raw data, do not present all of it in the results section. Instead, present only the information most appropriate to your audience's purpose in reading the document, summarizing other key information in graphs and figures. If appropriate, include your raw data in an appendix, referring to them within your text.
\end{quotation}

\lipsum[52-60]

\section{Summary (Optional)}



