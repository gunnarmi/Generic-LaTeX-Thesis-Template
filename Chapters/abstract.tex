
\chapter*{Abstract}
\addcontentsline{toc}{chapter}{Abstract} % DO NOT CHANGE 
\label{chap:abstract}

An abstract is a brief summarizing statement, not more than one page. It gives the reader a synopsis of the research problem, method, results, and conclusions of your document. The abstract takes the form of a single paragraph and should not contain cross references or citations. Abstracts are sometimes collected into volumes and must be able to stand alone. They may be read by parties trying to decide whether or not to read the main document, or for getting a broad picture before starting on the report. If you describe the content of each main chapter, and bind it nicely together, you’re done. You should not have any information in the abstract that is not found in any of the main chapters. It is common to close the abstract with a few well carefully selected keywords. Obviously, the abstract is the last thing you do in your project. 

Here is an example of a short and concise abstract \cite{winger12e3s}:

\begin{quotation}
\noindent  This thesis presents an evaluation of a set of 3D Scene Graph APIs for Java. The work consists mainly of two parts: Defining a methodology for comparing the APIs, and then applying the proposed methodology to the APIs.
An overview of the available 3D Scene Graph APIs in Java is presented, and a selection of these are chosen for the evaluation. The APIs subjected to the evaluation are Java 3D, Ardor3D and jMonkeyEngine3.
The proposed methodology focuses on the comparison on four different aspects. These are: \textit{Project Management and Technical Infrastructure, System Architecture, System Features and Capabilities, and System Performance}.
The results from applying the evaluation method show that none of the APIs were superior to the others in all respects. The results identify strengths and weaknesses with each API, that indicate which use cases each API might be better suited for.
\newline 
\paragraph{Keywords:} Scene Graph, API, Evaluation, Java, 3D Graphics, OpenGL, Java3D, jMonkeyEngine3, Ardor3D
\end{quotation}


